%Hey, if you're using this preamble it means that it was probably written by Stefano Graziosi (me). If you see something that doesn't make sense, feel free to email me at stefano.graziosi@studbocconi.it
%p.s. in case it's not already evident from the preamble, I'm not a professional LaTeX user, so I'm sure there are better ways to do things. I'm just trying to make it work.

% =============================================================================
%  Preamble maintained by Stefano Graziosi
%  If anything looks odd, ping: stefano.graziosi@studbocconi.it
%  LAST UPDATE: 18-01-2026
% =============================================================================

% I don't own copyright on anything; credits to original authors.

% =============================================================================
%  CORE PACKAGES (general utilities, fonts/encoding, colors, math, layout)
% =============================================================================

% Headers/footers
\usepackage{fancyhdr}

% Encoding & fonts
\usepackage[T1]{fontenc}
\usepackage{lmodern,mathrsfs}
\usepackage[table,xcdraw,dvipsnames]{xcolor}        % Colors (named colors like MidnightBlue, Cerulean, etc.)
\usepackage[many]{tcolorbox}                        % Boxed environments
\usepackage{graphicx}                               % Graphics
\usepackage{bookmark}                               % (Optional, improves bookmark handling)
\usepackage{enumitem}                               % For fancy enumerates
\usepackage{listings}                               % For the coding environment

% Hyperlinks (kept in original order to preserve behavior)
\usepackage{hyperref}
\hypersetup{
  pdftitle={\jobname},
  pdfauthor={Stefano Graziosi},
  colorlinks=true,
  linkcolor=MidnightBlue,
  citecolor=ForestGreen,
  urlcolor=sgpurple
}

% Math
\usepackage{amsmath, amssymb, amsthm}
\usepackage{mathtools, amsfonts, bm}
\usepackage{thmtools}

% Line spacing & multicolumn text
\usepackage{setspace,multicol}

% Title image & PDF inclusion
\usepackage{titlesec}
\usepackage{titlepic}
\usepackage{pdfpages}

% Quotes
\usepackage[version=4]{mhchem}
\usepackage{stmaryrd}
\usepackage{fvextra}
\usepackage{csquotes}

% Fun/extras
\usepackage{halloweenmath}
\usepackage{kantlipsum}

% =============================================================================
%  TABLES
% =============================================================================
\usepackage{array}
\usepackage{tabularx}
\usepackage{booktabs}
\usepackage{threeparttable}
\usepackage{multirow}
\usepackage{siunitx}
\sisetup{
    round-mode=places,
    round-precision=3,
    table-number-alignment=center
}

\usepackage[nameinlink,noabbrev]{cleveref}

\usepackage{codehigh}
\usepackage[normalem]{ulem}


% =============================================================================
%  FIGURES & SUBFIGURES
% =============================================================================
% (graphicx loaded above intentionally to keep original load order)
\usepackage{wrapfig}
\usepackage{float}
\usepackage{caption}
\captionsetup{font=small, labelfont=bf, labelsep=period}
\usepackage{subcaption}

% =============================================================================
%  DIAGRAMS AND PLOTS
% =============================================================================

\usepackage[all]{xy}
\usepackage{tikz}
\usetikzlibrary{calc,arrows.meta,positioning}
% (Optional speed-up for heavy docs; requires -shell-escape)
% \usetikzlibrary{external}
% \tikzexternalize[prefix=tikzcache/]
\usepackage[export]{adjustbox}
\usepackage{pgfplots}
\pgfplotsset{compat=1.18}
\usepgfplotslibrary{groupplots}


% =============================================================================
%  Equations
% =============================================================================

\numberwithin{equation}{section} % or \counterwithin{equation}{section}

% =============================================================================
%  PAGE GEOMETRY
% =============================================================================
\usepackage[a4paper,margin=1in]{geometry}
% \usepackage[margin=1in]{geometry} % (alt)

% =============================================================================
%  BIBLIOGRAPHY
% =============================================================================
\usepackage[backend=biber,
            style=authoryear,          % options: numeric, verbose, authoryear, authorname
            sorting=nyt]{biblatex}
\addbibresource{references.bib}
\usepackage{csquotes}               % smarter quotes & works great with biblatex



% =============================================================================
%  CHAPTER/SECTION FORMATTING
% =============================================================================

% Figure/Table numbering by section
\usepackage{chngcntr}
\counterwithin{figure}{section}
\counterwithin{table}{section}
\renewcommand{\thefigure}{\thesection-\arabic{figure}}
\renewcommand{\thetable}{\thesection-\arabic{table}}

% =============================================================================
%  COLORS (custom palette)
% =============================================================================

\definecolor{sgblue}{RGB}{0,169,211}
\definecolor{sggreen}{RGB}{0,164,0}
\definecolor{sgpurple}{RGB}{99,0,165}
\definecolor{sgyellow}{RGB}{255,211,0}
\definecolor{sgorange}{RGB}{255,127,20}
\definecolor{sbblue}{RGB}{219,248,254}
\definecolor{sbgreen}{RGB}{223,255,218}
\definecolor{sbpurple}{RGB}{241,220,255}

% =============================================================================
%  GENERIC BOX ENVIRONMENT
% =============================================================================

% Standard LaTeX box with color accents
\newtcolorbox{mybox}[3][]{
  colframe = #2!25,
  colback  = #2!10,
  coltitle = #2!20!black,
  title    = {#3},
  #1,
}

% =============================================================================
%  BASIC PROBLEM/SOLUTION ENVIRONMENTS
% =============================================================================

% "Problem" environment
\newtheorem{problem}{Problem}

% Personalized "Solution" environment
\newenvironment{solution}[1][\it{\textcolor{MidnightBlue}{Solution}}]{\textbf{#1. } }{\textcolor{MidnightBlue}{$\square$}}

% =============================================================================
%  THEOREM STYLES & BOXED PRESENTATION
% =============================================================================

% A helper length to fine-tune the label alignment.
% We measure the width of a single normal-space character, store it in \spacelength,
% and later shift the heading by exactly that width (negative hskip) to line things up.
\newlength{\spacelength}
\settowidth{\spacelength}{\normalfont\ }

% ----------------------------- STYLE: "theorem" ------------------------------
% This defines a thmtools STYLE named "theorem".
% Later, any environment declared as \declaretheorem[style=theorem]{<env>}
% will use the fonts/formatting set below for its HEAD LINE (the bold label area).
\declaretheoremstyle[
  headfont={\bfseries\sffamily\footnotesize},  % font for the heading (e.g., "Theorem 2")
  notefont={\normalfont},                      % font for the optional note (the [...] part)
  bodyfont={\normalfont},                      % font for the body text inside the environment
  headpunct={\relax},                          % punctuation after the heading; \relax = none
  % headformat controls how the label is typeset.
  % \NAME   -> the environment name ("Theorem", "Proposition", etc.)
  % \NUMBER -> the auto number (e.g., "2.3")
  % \NOTE   -> the optional note, provided by \begin{theorem}[<NOTE>]
  % \marginparsep is the standard gap used for margin notes; you reuse it as a spacing constant.
  % \makebox[0pt][r]{...} puts the label in a zero-width box, right-aligned,
  % and then \hskip-\spacelength nudges the following text left by one space width
  % so that the body aligns nicely under the heading.
  headformat={\makebox[0pt][r]{\NAME\ \NUMBER\hspace{\marginparsep}}\hskip-\spacelength{\normalsize\NOTE}},
]{theorem}

% ----------------- tcolorbox skin for any env using name "theorem" -----------
% This does NOT create an environment; it says: "whenever the 'theorem'
% environment is used, wrap its contents with this box styling."
\tcolorboxenvironment{theorem}{
  boxrule=0pt,                                % no outer rectangular frame
  boxsep=0pt,                                 % no internal padding around the content (we set our own left/right)
  colback={White!94!ForestGreen},             % Light forest green background
  enhanced jigsaw,                            % enables advanced tcolorbox features
  borderline west={1pt}{0pt}{ForestGreen},    % a 1pt vertical colored bar on the LEFT
  sharp corners,                              % squared corners (not rounded)
  before skip=10pt,                           % vertical space before the box
  after skip=10pt,                            % vertical space after the box
  left=5pt, right=5pt,                        % inner left/right padding (content inset)
  breakable,                                  % allow the box to split across pages
}

% --------------------------- PROPOSITION setup -------------------------------
% Create "proposition" as a thmtools theorem that USES the "theorem" STYLE above.
\declaretheorem[style=theorem]{proposition}

% And give "proposition" its own box skin (purple left border). Same mechanics as "theorem".
\tcolorboxenvironment{proposition}{
  boxrule=0pt,
  boxsep=0pt,
  colback={White!94!Mulberry},
  enhanced jigsaw,
  borderline west={1pt}{0pt}{Mulberry},
  sharp corners,
  before skip=10pt,
  after skip=10pt,
  left=5pt,
  right=5pt,
  breakable,
}

% ---------------------------- THEOREM setup ----------------------------------
% Explicitly declare the "theorem" environment (thmtools way) using the "theorem" style.
% Note: the box style for "theorem" is already defined above via \tcolorboxenvironment{theorem}{...}.
\declaretheorem[style=theorem]{theorem}

% ----------------------------- STYLE: "proof" --------------------------------
% amsthm already defines a proof environment, so we first neutralize it:
\let\proof\relax
\let\endproof\relax

% Now we declare a thmtools STYLE named "proof" for the heading of the proof environment.
\declaretheoremstyle[
  headfont={\small\scshape},                  % small caps for "Proof"
  notefont={\normalfont},
  bodyfont={\normalfont},
  headpunct={\relax},
  headformat={\makebox[0pt][r]{\NAME\hspace{\marginparsep}}\hskip-\spacelength{\NOTE}},
]{proof}

% And we hook the "proof" environment to a tcolorbox with a thin black left rule.
\tcolorboxenvironment{proof}{
  boxrule=0pt,
  boxsep=0pt,
  blanker,                                    % removes the typical tcolorbox background to look like plain text
  borderline west={1pt}{0pt}{black},          % thin black bar on the left
  before skip=10pt,
  after skip=10pt,
  left=5pt,
  right=5pt,
  breakable,
}

% Now we REDECLARE the actual "proof" environment using the "proof" STYLE.
% The 'qed=\qedsymbol' option ensures the end-of-proof □ is inserted automatically at the end.
\declaretheorem[style=proof, qed=\qedsymbol]{proof}

% ------------------------------ STYLE: "claim" -------------------------------
% A lighter-looking style used for "Intuition" and similar notes (italic header).
\declaretheoremstyle[
  headfont={\footnotesize\itshape},
  notefont={\normalfont},
  bodyfont={\normalfont},
  headpunct={\relax},
  headformat={\makebox[0pt][r]{\NAME\hspace{\marginparsep}}\hskip-\spacelength{\NOTE}},
]{claim}

% Create an "Intuition" environment that uses the 'claim' style (no special box here).
\declaretheorem[style=claim]{Intuition}

% -------------------- Environments declared the amsthm way -------------------
% Below you switch to amsthm’s API: \theoremstyle{<name>} then \newtheorem{...}.
% NOTE: \theoremstyle selects an amsthm style (e.g., plain/definition/remar k),
% not a thmtools style. You happen to use the same style NAME "theorem",
% but that name is defined via thmtools above—not via amsthm’s \newtheoremstyle.
% In many setups, \theoremstyle{theorem} has no effect unless you also defined an
% amsthm style called "theorem". Your boxes still appear because
% \tcolorboxenvironment{<envname>}{...} wraps by ENVIRONMENT NAME.

\theoremstyle{theorem}             % (May be a no-op unless an amsthm style "theorem" exists.)
\newtheorem{ques}{Question}        % Unboxed label styling depends on current amsthm style.

% Definition with its own box skin (Cerulean left rule). The header font/styling here
% comes from the current amsthm \theoremstyle (see the note above).
\theoremstyle{theorem}
\newtheorem{definition}{Definition}
\tcolorboxenvironment{definition}{
  boxrule=0pt,
  boxsep=0pt,
  colback={White!94!Cerulean},
  enhanced jigsaw,
  borderline west={1pt}{0pt}{Cerulean},
  sharp corners,
  before skip=10pt,
  after skip=10pt,
  left=5pt,
  right=5pt,
  breakable,
}

% Lemma with a Rhodamine bar
\theoremstyle{theorem}
\newtheorem{lemma}{Lemma}
\tcolorboxenvironment{lemma}{
  boxrule=0pt,
  boxsep=0pt,
  blanker,
  borderline west={1pt}{0pt}{Rhodamine},
  before skip=10pt,
  after skip=10pt,
  sharp corners,
  left=5pt,
  right=5pt,
  breakable,
}

% Remark with a BurntOrange bar
\theoremstyle{theorem}
\newtheorem{remark}{Remark}
\tcolorboxenvironment{remark}{
  boxrule=0pt,
  boxsep=0pt,
  colback={White!94!BurntOrange},
  enhanced jigsaw,
  borderline west={1pt}{0pt}{BurntOrange},
  before skip=10pt,
  after skip=10pt,
  sharp corners,
  left=5pt,
  right=5pt,
  breakable,
}

% Hint with a BurntOrange bar
\theoremstyle{theorem}
\newtheorem{hint}{Hint}
\tcolorboxenvironment{hint}{
  boxrule=0pt,
  boxsep=0pt,
  colback={White},
  enhanced jigsaw,
  borderline west={1pt}{0pt}{BurntOrange},
  before skip=10pt,
  after skip=10pt,
  sharp corners,
  left=5pt,
  right=5pt,
  breakable,
}

% Warning with a red bar
\theoremstyle{theorem}
\newtheorem{warning}{Warning}
\tcolorboxenvironment{warning}{
  boxrule=0pt,
  boxsep=0pt,
  colback={White},
  enhanced jigsaw,
  borderline west={1pt}{0pt}{red},
  before skip=10pt,
  after skip=10pt,
  sharp corners,
  left=5pt,
  right=5pt,
  breakable,
}

% Warning with a red bar
\theoremstyle{theorem}
\newtheorem{disclaimer}{Disclaimer}
\tcolorboxenvironment{disclaimer}{
  boxrule=0pt,
  boxsep=0pt,
  colback={White},
  enhanced jigsaw,
  borderline west={1pt}{0pt}{red},
  before skip=10pt,
  after skip=10pt,
  sharp corners,
  left=5pt,
  right=5pt,
  breakable,
}

% Question with a red bar
\theoremstyle{theorem}
\newtheorem*{question}{Question}
\tcolorboxenvironment{question}{
  boxrule=0pt,
  boxsep=0pt,
  colback={White},
  enhanced jigsaw,
  borderline west={1pt}{0pt}{red},
  before skip=10pt,
  after skip=10pt,
  sharp corners,
  left=5pt,
  right=5pt,
  breakable,
}

% Corollary with a WildStrawberry bar
\theoremstyle{theorem}
\newtheorem{corollary}{Corollary}
\tcolorboxenvironment{corollary}{
  boxrule=0pt,
  boxsep=0pt,
  enhanced jigsaw,
  borderline west={1pt}{0pt}{WildStrawberry},
  before skip=10pt,
  after skip=10pt,
  sharp corners,
  left=5pt,
  right=5pt,
  breakable,
}

% Example with a Dandelion bar
\theoremstyle{theorem}
\newtheorem{example}{Example}
\tcolorboxenvironment{example}{
  boxrule=0pt,
  boxsep=0pt,
  blanker,
  borderline west={1pt}{0pt}{Dandelion},
  before skip=10pt,
  after skip=10pt,
  sharp corners,
  left=5pt,
  right=5pt,
  breakable,
}

% Additional helpers using the 'claim' style (amsthm path).
% Again, \theoremstyle{claim} here refers to an amsthm style named "claim".
% You defined a thmtools style named "claim" above, which doesn't automatically
% register with amsthm. If no amsthm style "claim" exists, this line may be a no-op.
\theoremstyle{claim}
\newtheorem{intu}{Intuition}

\theoremstyle{claim}
\newtheorem{solu}{Solution}

% =============================================================================
%  CODE LISTINGS - GENERAL
% =============================================================================

% 1. Define your custom colors (Ensuring they are defined)
\definecolor{codegreen}{rgb}{0,0.6,0}
\definecolor{codegray}{rgb}{0.5,0.5,0.5}
\definecolor{codepurple}{rgb}{0.58,0,0.82}
\definecolor{backcolour}{rgb}{0.95,0.95,0.92}
% A specific color for Stata functions (to differentiate from commands)
\definecolor{statafunction}{rgb}{0.2,0.2,0.7} 

% 2. The Improved Stata Language Definition
\lstdefinelanguage{Stata}{
    sensitive=true,
    morecomment=[l]{//},
    morecomment=[s]{/*}{*/},
    morestring=[b]",
    morestring=[d]",
    morestring=[b]{`}{'},
    % Group 1: System/Flow (if, else, foreach, program, python)
    morekeywords=[1]{
        if, else, in, using, of, varlist, everything, foreach, forvalues, 
        while, continue, break, program, syntax, return, ereturn, sreturn, 
        args, capture, noisily, quietly, set, version, global, local, scalar, 
        matrix, mat, frame, python, java, putexcel, collect, do, run, ado, 
        clear, exit,
    },
    % Group 2: Data Manipulation (gen, egen, sort, merge)
    morekeywords=[2]{
        generate, gen, replace, egen, rename, ren, drop, keep, sort, bysort, 
        order, collapse, merge, append, reshape, encode, decode, destring, 
        tostring, recode, mvencode, mvdecode, use, save, import, export, 
        infile, infix, describe, desc, list, count, inspect, codebook, label, 
        notes, compress, format, cast,
    },
    % Group 3: Statistics (reg, sum, mixed, lasso, bayes)
    morekeywords=[3]{
        regress, reg, summarize, sum, tabulate, tab, logit, probit, tobit, 
        poisson, nbreg, xtreg, xtlogit, xtpoisson, xtset, xtdescribe, mixed, 
        meqrlogit, mepoisson, ivregress, areg, rreg, qreg, lasso, elasticnet, 
        bayes, bayesmh, meta, predict, predictnl, margins, marginsplot, test, 
        testnl, lincom, nlcom, contrast, pwcompare, estat, estimates, est, 
        pwcorr, correlate, corr, ttest, anova, oneway,
    },
    % Group 4: Graphics (twoway, scatter)
    morekeywords=[4]{
        graph, twoway, scatter, line, connected, bar, hbar, box, hbox, pie, 
        histogram, kdensity, function, title, subtitle, xtitle, ytitle, 
        legend, xlabel, ylabel, note, caption, saving,
    },
    % Group 5: Reserved/Types (byte, double, _n)
    morekeywords=[5]{
        byte, int, long, float, double, str, strL, c, _N, _n, _rc, _all, 
        aggregate, array, boolean, class, colvector, complex, const, delegate, 
        delete, eltypedef, enum, explicit, external, friend, inline, mata, 
        namespace, new, numeric, NULL, operator, orgtypedef, pointer, 
        polymorphic, pragma, private, protected, public, quad, real, 
        rowvector, short, signed, static, struct, super, switch, template, 
        this, throw, transmorphic, try, typedef, typename, union, unsigned, 
        vector, virtual, void, volatile,
    },
    % Groups 6-12: Functions (Dates, Math, Matrix, Strings)
    morekeywords=[6]{
        bofd, Cdhms, Chms, Clock, clock, Cmdyhms, Cofc, cofC, Cofd, cofd, 
        daily, date, day, dhms, dofb, dofC, dofc, dofh, dofm, dofq, dofw, 
        dofy, dow, doy, halfyear, halfyearly, hh, hhC, hms, hofd, hours, 
        mdy, mdyhms, minutes, mm, mmC, mofd, month, monthly, msofhours, 
        msofminutes, msofseconds, qofd, quarter, quarterly, seconds, ss, 
        ssC, tC, tc, td, th, tm, tq, tw, week, weekly, wofd, year, yearly, 
        yh, ym, yofd, yq, yw,
    },
    morekeywords=[7]{
        abs, ceil, cloglog, comb, digamma, exp, expm1, floor, invcloglog, 
        invlogit, ln, ln1m, ln1p, lnfactorial, lngamma, log, log10, log1m, 
        log1p, logit, max, min, mod, reldif, round, sign, sqrt, trigamma, 
        trunc, rbeta, rbinomial, rcauchy, rchi2, rexponential, rgamma, 
        rhypergeometric, rigaussian, rlaplace, rlogistic, rnbinomial, 
        rnormal, rpoisson, rt, runiform, runiformint, rweibull, rweibullph, 
        tin, twithin,
    },
    morekeywords=[8]{
        cholesky, coleqnumb, colnfreeparms, colnumb, colsof, det, diag, 
        diag0cnt, el, get, hadamard, I, inv, invsym, issymmetric, J, 
        matmissing, matuniform, mreldif, nullmat, roweqnumb, rownfreeparms, 
        rownumb, rowsof, sweep, trace, vec, vecdiag,
    },
    morekeywords=[9]{
        abbrev, char, collatorlocale, collatorversion, indexnot, plural, 
        regexm, regexr, regexs, soundex, soundex_nara, strcat, strdup, 
        string, stritrim, strlen, strlower, strltrim, strmatch, strofreal, 
        strpos, strproper, strreverse, strrpos, strrtrim, strtoname, strtrim, 
        strupper, subinstr, subinword, substr, tobytes, uchar, udstrlen, 
        udsubstr, uisdigit, uisletter, ustrcompare, ustrcompareex, ustrfix, 
        ustrfrom, ustrinvalidcnt, ustrleft, ustrlen, ustrlower, ustrltrim, 
        ustrnormalize, ustrpos, ustrregexm, ustrregexra, ustrregexrf, 
        ustrregexs, ustrreverse, ustrright, ustrrpos, ustrrtrim, 
        ustrsortkey, ustrsortkeyex, ustrtitle, ustrto, ustrtohex, 
        ustrtoname, ustrtrim, ustrunescape, ustrupper, ustrword, 
        ustrwordcount, usubinstr, usubstr, word, wordbreaklocale, wordcount,
    },
    morekeywords=[10]{
        acos, acosh, asin, asinh, atan, atanh, cos, cosh, sin, sinh, tan, 
        tanh,
    },
    morekeywords=[11]{
        betaden, binomial, binomialp, binomialtail, binormal, cauchy, 
        cauchyden, cauchytail, chi2, chi2den, chi2tail, dunnettprob, 
        exponential, exponentialden, exponentialtail, F, Fden, Ftail, 
        gammaden, gammap, gammaptail, hypergeometric, hypergeometricp, 
        ibeta, ibetatail, igaussian, igaussianden, igaussiantail, 
        invbinomial, invbinomialtail, invcauchy, invcauchytail, invchi2, 
        invchi2tail, invdunnettprob, invexponential, invexponentialtail, 
        invF, invFtail, invgammap, invgammaptail, invibeta, invibetatail, 
        invigaussian, invigaussiantail, invlaplace, invlaplacetail, 
        invlogistic, invlogistictail, invnbinomial, invnbinomialtail, 
        invnchi2, invnF, invnFtail, invnibeta, invnormal, invnt, invnttail, 
        invpoisson, invpoissontail, invt, invttail, invtukeyprob, 
        invweibull, invweibullph, invweibullphtail, invweibulltail, laplace, 
        laplaceden, laplacetail, lncauchyden, lnigammaden, lnigaussianden, 
        lniwishartden, lnlaplaceden, lnmvnormalden, lnnormal, lnnormalden, 
        lnwishartden, logistic, logisticden, logistictail, nbetaden, 
        nbinomial, nbinomialp, nbinomialtail, nchi2, nchi2den, nchi2tail, 
        nF, nFden, nFtail, nibeta, normal, normalden, npnchi2, npnF, npnt, 
        nt, ntden, nttail, poisson, poissonp, poissontail, t, tden, ttail, 
        tukeyprob, weibull, weibullden, weibullph, weibullphden, 
        weibullphtail, weibulltail,
    },
    morekeywords=[12]{
        autocode, byteorder, _caller, chop, clip, cond, e, fileexists, 
        fileread, filereaderror, filewrite, fmtwidth, has_eprop, inlist, 
        inrange, irecode, maxbyte, maxdouble, maxfloat, maxint, maxlong, mi, 
        minbyte, mindouble, minfloat, minint, minlong, missing, r, recode, 
        replay, smallestdouble,
    }
}

% 3. Your Style (Updated to include the new groups)
\lstdefinestyle{mystyle}{
    backgroundcolor=\color{backcolour},
    commentstyle=\color{codegreen},
    stringstyle=\color{codepurple},
    numberstyle=\tiny\color{codegray},
    basicstyle=\ttfamily\footnotesize,
    breakatwhitespace=false,
    breaklines=true,
    captionpos=b,
    keepspaces=true,
    numbers=left,
    numbersep=5pt,
    showspaces=false,
    showstringspaces=false,
    showtabs=false,
    tabsize=2,
    % --- KEYWORD MAPPING ---
    % Groups 1-4 are COMMANDS (regress, gen, etc.) -> Use your Magenta
    keywordstyle=[1]\color{magenta},
    keywordstyle=[2]\color{magenta},
    keywordstyle=[3]\color{magenta},
    keywordstyle=[4]\color{magenta},
    % Group 5 are TYPES (float, int) -> Use Blue or Magenta (User choice)
    keywordstyle=[5]\color{blue}, 
    % Groups 6-12 are FUNCTIONS (sum(), max()) -> Use a distinct color 
    % (Using Magenta here looks confusing as they look like commands). 
    % I selected 'statafunction' (defined above) which is a nice blue/purple.
    keywordstyle=[6]\color{statafunction},
    keywordstyle=[7]\color{statafunction},
    keywordstyle=[8]\color{statafunction},
    keywordstyle=[9]\color{statafunction},
    keywordstyle=[10]\color{statafunction},
    keywordstyle=[11]\color{statafunction},
    keywordstyle=[12]\color{statafunction},
}

\lstset{style=mystyle}
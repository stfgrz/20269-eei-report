\subsection{Empirical Results for Sector 13: Manufacture of Textiles}

        NACE 13 (Manufacture of Textiles) is a relatively labour-intensive industry. Production involves significant labour input (for tasks like spinning, weaving, finishing) alongside capital equipment (looms, knitting machines, etc.), but the labour component tends to be high. The data comprise firms from Spain, France, and Italy in this sector, pooled with country and year fixed effects included.

        \subsubsection{Labour Coefficient across Methods}

            The OLS estimate of the labour coefficient in the textile sector is likely biased due to the reasons discussed above. Indeed, the OLS coefficient on labour in sector 13 is found to be significantly different from the estimates produced by the Wooldridge and LP methods. OLS suggests a certain elasticity of output with respect to labour (for instance, hypothetically around 0.5–0.6 in log terms), but this estimate is suspect because it conflates true labour productivity with unobserved TFP shocks. After applying the Wooldridge proxy method, the labour elasticity in textiles increases noticeably – indicating that OLS had underestimated labour’s contribution. The Levinsohn–Petrin estimate of the labour coefficient is in a similar range as Wooldridge’s, reinforcing this correction. This pattern is consistent with the presence of simultaneity bias that attenuated the OLS labour coefficient. Intuitively, textile firms experiencing positive productivity shocks within the year likely boosted their usage of materials (dyes, fibers) and perhaps some labour overtime, but not one-for-one with output; OLS, failing to account for the shock, attributed relatively less output to labour. The control-function approaches recover a higher labour coefficient, suggesting that when unobserved productivity is properly controlled, labour input has a larger output elasticity than the naive OLS implied. In sum, for sector 13, \(\beta_L\) (labour) is higher in the WRDG and LP estimates compared to OLS, reflecting a correction of OLS’s bias and aligning with the labour-intensive nature of textiles.

        \subsubsection{Capital Coefficient across Methods}

            The estimated output elasticity with respect to capital in textiles (the capital coefficient \(\beta_K\)) shows a different pattern. OLS may either undervalue or overvalue capital’s role depending on how capital investment correlates with productivity. In labour-intensive textiles, one might expect the capital coefficient to be modest (perhaps on the order of 0.1–0.3). The OLS estimate in this sector likely suffered from endogeneity as well: more productive textile firms tend to expand capacity (increase capital stock) over time, which can induce a downward bias in OLS’s capital coefficient if these high-capital firms have high output mostly due to productivity advantage rather than capital alone. The WRDG and LP methods adjust for this by accounting for the productivity term. As a result, the capital coefficient in sector 13 under WRDG/LP is slightly higher than the OLS estimate. For example, if OLS estimated a capital elasticity around 0.10, the LP method might find it to be around 0.15–0.20. This upward adjustment implies that OLS had been undervaluing capital’s true effect, likely because the most capital-intensive firms were also the most productive and OLS could not disentangle these effects. Both Wooldridge and LP yield similar capital coefficients for textiles, and both are larger than the OLS coefficient, though capital remains quantitatively less important than labour in this sector. Importantly, the sum of the labour and capital elasticities in textiles remains below 1 in all methods (given that materials and other inputs also contribute), so constant returns to scale are not violated; the changes from OLS to LP/WRDG primarily redistribute explanatory power between labour and capital. Overall, in sector 13, OLS underestimates the labour elasticity and slightly underestimates capital’s elasticity; the advanced methods correct these, confirming that textiles production is very labour-driven but with a non-negligible capital contribution.

    \subsection{Empirical Results for Sector 29: Manufacture of Motor Vehicles}

        NACE 29 (Manufacture of Motor Vehicles, Trailers, etc.) is a capital-intensive industry. Automobile production relies on large-scale assembly plants, robotics, and heavy machinery, alongside skilled labour. Material inputs (components, steel, etc.) are also a significant part of the production process. The sample includes auto manufacturing firms from Spain, France, and Italy, with country and year effects controlled.

        \subsubsection{Labour Coefficient across Methods}

             In the automotive sector, the OLS-estimated labour coefficient is typically lower than that in textiles, reflecting the lesser labour intensity. Suppose OLS finds the labour output elasticity to be relatively small (e.g. \(\approx\)0.2–0.3), consistent with the idea that much of the output is driven by capital and intermediate inputs. However, this OLS labour coefficient may again be biased by simultaneity. If auto firms facing positive productivity shocks do not adjust employment as quickly (due to production schedules or rigid labour contracts) and instead adjust other inputs (like materials usage), OLS could either overestimate or underestimate labour’s effect. The Wooldridge and LP estimates for labour in sector 29 show an adjustment similar to sector 13 but potentially less dramatic. After controlling for unobserved productivity, the labour coefficient in sector 29 increases modestly relative to OLS (if OLS was downward biased), or in some cases could decrease if OLS had been upward biased. In practice, we observe that WRDG and LP yield a slightly higher labour elasticity than OLS for motor vehicles, though the difference is not as large as in textiles. For instance, labour’s elasticity might rise from \(\approx\)0.25 (OLS) to \(\approx\)0.30–0.35 with LP. This suggests OLS likely understated labour’s role even in this capital-intensive sector, though labour remains less influential than in textiles. The fact that the bias correction is smaller here could indicate that in automotive manufacturing, contemporaneous adjustment of labour to shocks is less flexible (firms rely more on adjusting materials or utilize excess capacity of existing labour), so the simultaneity bias affecting labour is less severe than in sector 13.

        \subsubsection{Capital  Coefficient across Methods}

            The capital coefficient in sector 29 is expected to be relatively high, reflecting the industry’s heavy reliance on machinery and equipment. OLS might estimate a capital elasticity on the order of 0.3 or higher for autos. However, simultaneity and selection issues can distort this. A common bias in OLS is to underestimate the capital coefficient because the most capital-intensive firms are also those with high productivity (thus high output not solely due to capital). Moreover, firms with persistently low productivity might exit, meaning OLS disproportionately uses surviving high-capital firms in estimation, further complicating the inference of capital’s true effect. The WRDG and LP methods correct for these issues. In the automotive data, both methods indeed deliver a higher capital coefficient than OLS. For example, if OLS gave \(\beta_K \approx 0.30\), LP might find \(\beta_K \approx 0.40\). The Wooldridge estimate is in close agreement with LP, perhaps \(\beta_K \approx 0.38-0.42\), indicating a robust result. This increase confirms that OLS had been biased downward for capital: once we control for the unobserved productivity (which OLS was partly attributing to “luck” rather than capital), we see that capital investment has an even larger impact on output than initially thought. Notably, in sector 29 the sum of labour and capital coefficients is substantially higher than in textiles – consistent with motor vehicles production having higher returns to scale when including these two inputs (though if materials were included, the production function likely exhibits approximately constant returns overall). The capital intensity of sector 29 is evident in all methods, but WRDG/LP accentuate it further by correcting OLS’s bias. In summary, for sector 29, \(\beta_L\) is relatively small (and slightly larger under WRDG/LP than OLS), while \(\beta_K\)is large and significantly higher in the bias-corrected methods than under OLS. This aligns with prior expectations that OLS undervalues capital’s role in a capital-intensive industry due to endogeneity and selection, which the advanced estimators successfully address.

    \subsection{Comparison Across Methods and Sectors}

        Examining the patterns across the two industries and three estimation methods, several systematic differences emerge:

            \paragraph{Bias in OLS vs.\ Corrected Methods}

                In both sectors, OLS appears to yield biased coefficients for inputs. The Wooldridge and Levinsohn–Petrin methods produce higher labor and capital coefficients in most cases, suggesting that OLS was understating these elasticities due to simultaneity bias. This is especially pronounced for the labor coefficient in the textile sector (NACE 13), where the gap between OLS and WRDG/LP estimates is larger, consistent with a higher susceptibility to endogeneity in a labor-intensive setting. In the automotive sector, the bias corrections (differences between OLS and WRDG/LP) are present but somewhat smaller, indicating that while simultaneity matters, its impact on OLS estimates was less extreme in that context.

            \paragraph{Labor vs.\ Capital Intensity}

                The two sectors show clear differences in the relative magnitude of labor and capital coefficients, reflecting their production technologies. Sector 13 (textiles) consistently exhibits a higher labor elasticity and a lower capital elasticity than sector 29 under all estimation methods, confirming that textiles production relies more on labor input. Sector 29 (motor vehicles) shows the opposite: a higher capital elasticity and lower labor elasticity, highlighting its capital-intensive nature. These differences are robust across OLS, WRDG, and LP, indicating intrinsic technological differences between the industries. The advanced estimation methods reinforce these distinctions—for instance, after correcting biases, the labor coefficient in textiles remains substantially above the labor coefficient in autos, and the capital coefficient in autos remains well above that in textiles.

            \paragraph{Wooldridge vs.\ Levinsohn–Petrin}

                The WRDG and LP estimates are generally consistent with each other. Both methods control for the unobserved productivity and yield similar coefficients for labor and capital in each sector. Any differences between the WRDG and LP estimates are minor—for example, one might find LP gives a slightly higher labor coefficient than WRDG in one sector, but a slightly lower in another, without a clear systematic gap. This suggests that the choice of proxy method (one-step GMM vs.\ two-step semi-parametric) does not drastically alter the resulting elasticities, which is reassuring. It also implies that the production function coefficients obtained are not sensitive to the specific estimation technique as long as simultaneity is addressed; both approaches successfully mitigate endogeneity bias and converge on a comparable portrayal of the production technology. \\
        
        In light of these comparisons, we see a coherent story: OLS, which ignores the endogeneity of input choice, underestimates the contribution of both labor and capital in these production functions. Methods like Wooldridge’s and Levinsohn–Petrin’s, which account for unobserved productivity (a key source of simultaneity bias), deliver higher and more credible input elasticities. The corrections are most notable for the input that is flexibly adjusted (labor in textiles, materials in general) and for capital in a setting where investment correlates with firm survival and productivity (automotive). Across sectors, the fundamental differences in production structure (labor-intensive vs.\ capital-intensive) are reflected in the magnitude of coefficients and remain evident after proper estimation. This consistency lends confidence that the WRDG and LP estimates are capturing true technological parameters, whereas OLS was clouded by bias.

    \subsection{Controlling for Country and Year Effects}

        All the above estimations include country fixed effects and year fixed effects. This means dummy variables for each country (Spain, France, Italy) and each year in the panel were incorporated as additional regressors (or equivalently, data were demeaned by country and year). Controlling for country effects addresses cross-country heterogeneity: each country may have a different baseline level of productivity, institutional environment, or measurement units that affect output. 
        
         A positive year effect might capture, for instance, a year where all firms experienced higher output (perhaps due to a demand boom or a policy change), while a negative year effect could capture a recession year. Including year dummies means that the estimation focuses on deviations from the overall yearly trends. This ensures that the labour and capital coefficients are not conflated with time-related growth in productivity. In other words, any aggregate improvement in productivity over time is absorbed by the year effects, while the coefficients \(\beta_L\) and \(\beta_K\)reflect the within-year, within-country relationship between inputs and output. By accounting for year effects, we acknowledge that productivity is not constant over time and allow for business-cycle influences on output that are independent of input usage. 
        
        Together, the country and year controls significantly enhance the rigour of the analysis. They address omitted-variable bias stemming from unobserved country-level factors and macroeconomic shocks. The result is that the estimated labour and capital elasticities are more likely to be purely capturing the production technology rather than artifacts of country differences or time trends. Indeed, the inclusion of these fixed effects did not qualitatively change the comparison across methods (OLS vs. WRDG vs. LP) discussed above, but it gives us greater confidence that we are comparing like with like across countries and years. Essentially, we estimate a “common” production function for each sector that is valid across Spain, France, and Italy after controlling for each country’s fixed advantage or disadvantage, and stable over the period after accounting for yearly fluctuations. This approach strengthens the validity of cross-country comparisons and ensures that the findings (e.g. the higher capital elasticity in autos) are not driven by one country’s data or a particular period.







**Simultaneity bias**


---

**Selection bias**


---

**How to overcome the issue**

| Estimator                   | How it tackles simultaneity                                                                                                                               | How it tackles selection                                                                                                       | Practical notes                                                                                                                                                                                                        |
| --------------------------- | --------------------------------------------------------------------------------------------------------------------------------------------------------- | ------------------------------------------------------------------------------------------------------------------------------ | ---------------------------------------------------------------------------------------------------------------------------------------------------------------------------------------------------------------------- |
| **Olley–Pakes (1996)**      | Uses investment as a monotonic proxy for $\omega_{it}$ in a control-function framework.                                                                   | Explicit survival equation conditions on the continuation probability.                                                         | Investment may be lumpy/zero for many firms.                                                                                                                                                                           |
| **Levinsohn–Petrin (2003)** | Replaces investment with intermediate inputs (materials) as the proxy; avoids zeros and exploits richer variation.                                        | Treats selection as second-order or adds an OP-style survival term if exit is important .                                      | Implemented in Stata’s `levpet` routine; requires choosing a polynomial order .                                                                                                                                        |
| **Wooldridge (2009)**       | Stacks all moment conditions in a single-step GMM system: the proxy variable deals with simultaneity while lagged inputs serve as additional instruments. | The survival-probability term (à la OP) can be included in the same GMM set, so selection is handled without a separate stage. | Computationally fast, delivers robust standard errors, and is less sensitive to the Ackerberg-Caves-Frazer labour-identification critique. Cited in the slide pack’s reference list as the preferred modern approach . |

**Why Wooldridge often wins in practice**

1. **Efficiency** – single-step GMM uses all available instruments at once, reducing finite-sample noise compared with the two-step LP/OP procedures.
2. **Robust standard errors** – the sandwich form comes “for free”, whereas LP/OP usually require bootstrapping.
3. **Flexible on selection** – including the OP survival-probability term is straightforward, so the same estimator simultaneously addresses both sources of bias.
4. **Light computational burden** – no high-order polynomials or kernel smoothing; suitable for large unbalanced panels common in firm-level datasets.

**Practical checklist**

* Always compare OLS with a proxy-variable estimator; a large drop in input elasticities signals severe simultaneity.
* If exit rates exceed, say, 5 % per annum, include a survival equation (OP or WRDG-style).
* Test instrument strength: the proxy (materials or investment) must vary within firms; weak proxies re-introduce endogeneity.
* For robustness, replicate results with alternative lags or the Ackerberg-Caves-Frazer modification that places current labour in the second-stage moments when identification is delicate.

Following these guidelines ensures that the estimated labour and capital coefficients—and all downstream objects such as firm-level TFP and mark-ups—are not contaminated by simultaneity or selection biases.
